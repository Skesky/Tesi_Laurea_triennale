La realizzazione di questo simulatore per la distribuzione quantistica di chiavi con modulazione gaussiana di prefigge lo scopo di realizzare una chiave crittografica sicura, la quale successivamente potr\`a essere utilizzate per la codifica e decodifica di un messaggio trasmesso su un canale classico. 

Come tecnologie \`e stato utilizzato il linguaggio di programmazione C++. I motivi principali di questa scelta sono tre: la velocit\`a del linguaggio di programmazione, la disponidilit\`a di librerie che implementano le distribuzioni di probabilit\`a necessarie alla simulazione e la presenza di librerie esterne che implementano due algoritmi indispensabili per la correzzione degli errori.

Anche se il progetto \`e stato realizzato utilizzando C++ \`e stato deciso di illustrare quanto \`e stato fatto attraverso dello psdeudo-codice in modo tale da non legare la simulazione ad una specifico linguaggio di programmazione ed aggevolarne la rimplementazione in qualunque linguaggio di programmazione si voglia utilizzare, di fatto verr\`a utilizzato della pseudo-codice anche per descrivere il due algoritmi per la correzzione degli errori.

\section{Preparazione, trasmissione e misura}

\subsection{Preparazione dello stato coerente}
Vengono scelti due valori che rappresentano i valori medi delle gaussiane che realizzano lo stato coerente. I valori vengono estratti a loro volta da una distribuzione di probabilit\`a normale centratati in zero e con varianza $V_{mod}$ che un questo specifica simulazione ha valore $5.226$. Questo valore di varianza \`e stato scelto, insieme al valore della trasmittanza e quello del rumore, per raggiungere un rapporto segnale rumore pari a $SNR=0.52$. 
\subsection{Trasmissione su canale rumoroso}
Per ogni stato coerente estratto si simula la trasmissione in un canale rumoroso. In particolare i valori di rumore e trasmittanza scelti sono rispettivamente $\xi = 0.005$ e $T = 0.1$, scelti sempre per garantire un $SNR = 0.52$
\subsection{Misura omodina del segnale}

\section{Stima dei parametri}

\subsection{Sifting}
\subsection{Stima delle varianze dei dati trasmessi, ricevuti e delle loro covarianza}
\subsection{Stima mutue informazioni}
\subsection{Accertamento sicurezza trasmissione}

\section{Riconciliazione}

\subsection{Generazione stringa random di bit}

\subsection{Algoritmo Progessive Edge Growth}


\begin{algorithm}
\caption{: Classic PEG algorithm}
\begin{algorithmic}
\STATE pippo 
\end{algorithmic}
\end{algorithm}


\begin{algorithm}
\caption{: Sum-product decoding algorithm}
\begin{algorithmic}
\STATE pippo
\end{algorithmic}
\end{algorithm}

\subsection{Messaggio fittizio}

\subsection{Algoritmo Sum-Product}

\begin{algorithm}
\caption{: Sum-product decoding algorithm}
\begin{algorithmic}
\STATE pippo
\end{algorithmic}
\end{algorithm}




