In questo elaborato abbiamo realizzato una simulazione del protocollo di distribuzione quantistica di chiavi a variabili continue con modulazione gaussiana. In principio abbiamo osservato una piccola introduzione storica della crittografia per poi andare a descrivere gli stati coerenti e le loro misure. Successivamente abbiamo trattato pi\`u nel dettaglio il protocollo scelto fino ad arrivare a come la simulazione \`e stata realizzata.

Infine, a causa dell'avvento dei computer quantistici, i quali sono una minaccia per i classici crittosistemi, abbiamo voluto mostrare come fosse possibile implementare un protocollo di crittografia che non risente di questa minaccia e come fosse possibile produrre una chiave crittografica teoricamente sicura.

La sicurezza qui descritta, \`e solamente teorica, perch\'e in una reale implementazione del protocollo c'\`e da tener conto anche dei dispositivi necessari per l'implementazione dello stesso e delle loro imperfezioni, quindi nella realt\`a \`e impossibile raggiungere una sicurezza assoluta. In pratica si pu\`o ottenere una sicurezza entro certi limiti che sono dati dalla realizzazione dei dispositivi fisici necessari per l'implementazione dei protocolli.

In merito alla realizzazione di dispositivi, nel prossimo futuro saranno disponibili delle certificazioni che attestano il reale livello di sicurezza che un certo dispositivo pu\`o garantire in una reale applicazione di un protocollo.

\break
Inoltre, tenendo conto delle dfficolt\`a odierne relative al rumore sul canale adoperato, la commercializzazione e la diffusione di calcolatori quantistici permetter\`a lo sviluppo e la propagazione nell'utilizzo di nuovi protocolli e di schemi di correzione dell'errore pi\`u performanti.



