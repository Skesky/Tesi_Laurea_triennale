In questo elaborato abbiamo realizzato una simulazione del protocollo di distribuzione quantistica di chiavi a variabili continue con modulazione gaussiana. Come prima cosa abbiamo fatto una piccola introduzione storica della crittografia per poi andare a descrivere gli stati coerenti e le loro misure. Successivamente abbiamo parlato pi\`u nel dettaglio del protocollo scelto fino ad arrivare a come la simulazione \`e stata realizzata.

A causa dell'avvento dei computer quantistici, i quali sono una minaccia per i classici crittosistemi, abbiamo voluto mostrare come fosse possibile implementare un protocollo di crittografia che non risente di questa minaccia e come fosse possibile produrre una chiave crittografica teoricamente sicura.

La sicurezza per\`o, \`e solamente teorica, perch\'e in una reale implementazione del protocollo c'\`e da tener conto anche dei dispositivi necessari per l'implementazione e delle loro imperfezioni, quindi nella realt\`a \`e impossibile raggiungere una sicurezza assoluta. Si pu\`o ottenere una sicurezza entro certi limiti che sono dati dalla realizzazione dei dispositivi fisici per la realizzazione dei protocolli.


